%!TEX program=luatex
% Falls ein 4:3 Format gewünscht ist, einfach `aspectratio=160` wegmachen
\documentclass[8pt,xcolor={dvipsnames},aspectratio=169]{beamer}
% Preamble (((
\usepackage[T1]{fontenc}    % sorgt dafür, dass wir alle möglichen Zeichen in unserer Font drinhaben
\usepackage{polyglossia}    % erlaubt lualatex sprachabhängige dinge zu machen. Worttrennungen etc. 
\usepackage{csquotes}       % erlaubt den \enqoute{} Befehl, um korrekte Anführungszeichen um ein Textblock zu machen
\usepackage[backend=biber, style=alphabetic]{biblatex} % Literaturverzeichnis 
%\setdefaultlanguage[spelling=new]{german} % wir optimieren auf Deutschen Text
%\setdefaultlanguage{english} % wir optimieren auf englischen Text
\setdefaultlanguage[spelling=new]{german} % wir optimieren auf deutschen Text
\usetheme[%
    progressbar=frametitle,% Der Fortschrittsbalken in der Überschrift
    sectionpage=progressbar% Zwischen Sections gibt es auch nochmal einen Fortschrittsbalken
    ]{metropolis} % Das theme metropolis sieht halt nice aus
\usefonttheme[onlymath]{serif} % Normalerweise wird Text in beamer in sans gesetzt. Auch Mathe. Das mag ich nicht...
\usepackage{tikz}
\usetikzlibrary{positioning}
\usepackage{tcolorbox}
\usepackage{proof}
% )))
% ---------------------------- Viele Einstellungen für das Theme ---------------- (((
\definecolor{tugreen}{RGB}{132,184,24}
\definecolor{backgroundstuff}{RGB}{34,55,58}
\setbeamercolor*{frametitle}{bg=tugreen, fg=white}
\setbeamercolor{progress bar}{fg=tugreen, bg=tugreen!50!black}
\setbeamercolor{block title default}{bg=tugreen!80!black, fg=white}
\setbeamercolor{block body default}{bg=tugreen!20, fg=black}
\setbeamercolor{block title example}{bg=black!60, fg=white}
\setbeamercolor{block body example}{bg=black!90, fg=white}
\setbeamercolor{block title alert}{bg=red!60, fg=white}
\setbeamercolor{block body alert}{bg=red!90, fg=white}
\makeatletter
\setlength{\metropolis@titleseparator@linewidth}{2pt}
\setlength{\metropolis@progressonsectionpage@linewidth}{2pt}
\setlength{\metropolis@progressinheadfoot@linewidth}{2pt}
\metroset{block=fill}
\newcommand{\var}{\textbf{\tiny(Var)}}
\newcommand{\app}{\textbf{\tiny(App)}}
\newcommand{\abs}{\textbf{\tiny(Abs)}}
\renewcommand{\implies}{\Rightarrow}
\makeatother
\setbeamertemplate{itemize item}[square]
\setbeamertemplate{itemize subitem}[circle]
\setbeamercolor{itemize item}{fg=tugreen}
\setbeamercolor{itemize subitem}{fg=tugreen}
\setbeamerfont{description item}{series=\bfseries}
\setbeamercolor{description item}{fg=tugreen}
\tikzset{
    invisible/.style={opacity=0,text opacity=0},
    visible on/.style={alt={#1{}{invisible}}},
    alt/.code args={<#1>#2#3}{%
      \alt<#1>{\pgfkeysalso{#2}}{\pgfkeysalso{#3}} % \pgfkeysalso doesn't change the path
    },
  }

% ))) -------------------------------------------------------------------------------
% Customizations (((
\date{\today} % Das Datum des Vortrags\
\bibliography{main} % Einbinden der Bibliography (falls vorhanden): main.bib

\title{Titel des Vortrags}
\subtitle{Untertitel}
\author[V. Nachname]{Vorname Nachname}
\institute{Miscatonic University\\Arkham, MA\\Faculty for the occult} % Oder was auch immer da reinkommt
% )))
\begin{document}
\frame{\maketitle} % Einen Frame für den Titel
% Inhaltsverzeichnis (((
\begin{frame}{Inhalt} 
  \setbeamertemplate{section in toc}[sections numbered]
\tableofcontents[hideallsubsections]
\end{frame}
% )))
% Itemizes und Enumerations (((
\section{Itemizes und Enumerations}
\subsection{Itemize}
\begin{frame}{Itemize}    
    \begin{itemize}
        \item Item
        \item[$\Rightarrow$] Item
        \item Item
            \begin{itemize}
                \item Subitem
                \item Subitem
            \end{itemize}
    \end{itemize}
\end{frame}
\subsection{Enumeration}
\begin{frame}{Enumeration}    
    \begin{enumerate}
        \item Item
        \item Item
        \item Item 
            \begin{enumerate}
                \item Subitem
                \item Subitem
            \end{enumerate}
    \end{enumerate}
\end{frame}
\subsection{Description}
\begin{frame}{Description}    
    \begin{description}
        \item[Desc:] Item
        \item[Desc:] Item
        \item[Desc:] Item 
    \end{description}
\end{frame}
% )))
% Aufteilungen (((
\section{Aufteilungen}
\begin{frame}{Aufteilungen} 
    \begin{columns}
        \column{.49\textwidth}
    \begin{itemize}
        \item Liste
        \item Auf
        \item Der
        \item Einen 
        \item Seite
    \end{itemize}
        \column{.49\textwidth}
        \begin{figure}[htpb]
        \begin{center}
        \begin{tikzpicture}
            \draw (0,0) circle (2cm);
            \draw (-.75,.75) circle (0.1cm);
            \draw (.75,.75) circle (0.1cm);
            \draw[bend right=20] (-.75,-.75) to (.75,-.75);
        \end{tikzpicture}
        \end{center}
        \caption{Smile auf der anderen}%
        \end{figure}        
\end{columns}
\end{frame}
% )))
% Boxen (((
\begin{frame}{Boxen}
    \begin{tcolorbox}[title=Eine Box]
        Boxen sind cool, siehe\cite{colorman}
    \end{tcolorbox}
\end{frame}
% )))
% Literatur (((
\nocite{*}
\section{References}
\begin{frame}[allowframebreaks]{References}
\printbibliography[heading=none]
\end{frame}
% )))
\end{document}

% vim:set foldmethod=marker foldmarker=(((,))) spelllang=de_de:
